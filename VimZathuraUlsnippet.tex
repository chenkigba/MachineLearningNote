\documentclass[UTF_8]{ctexart}
\usepackage{graphicx, amsfonts, setspace, listings, color, xcolor}

\begin{document}

\definecolor{dkgreen}{rgb}{0,0.6,0}
\definecolor{gray}{rgb}{0.5,0.5,0.5}
\definecolor{mauve}{rgb}{0.58,0,0.82}
\definecolor{ltgray}{HTML}{E6E6E6} 
\lstset{ %
language=python,                % the language of the code
basicstyle=\footnotesize,       % the size of the fonts that are used for the code
numbers=left,                   % where to put the line-numbers
numberstyle=\tiny\color{gray},  % the style that is used for the line-numbers
stepnumber=2,                   % the step between two line-numbers. If it's 1, each line 
                                % will be numbered
numbersep=5pt,                  % how far the line-numbers are from the code
backgroundcolor=\color{ltgray},  % choose the background color. You must add \usepackage{color}
showspaces=false,               % show spaces adding particular underscores
showstringspaces=false,         % underline spaces within strings
showtabs=false,                 % show tabs within strings adding particular underscores
frame=single,                   % adds a frame around the code
rulecolor=\color{black},        % if not set, the frame-color may be changed on line-breaks within not-black text (e.g. commens (green here))
tabsize=2,                      % sets default tabsize to 2 spaces
captionpos=b,                   % sets the caption-position to bottom
breaklines=true,                % sets automatic line breaking
breakatwhitespace=false,        % sets if automatic breaks should only happen at whitespace
title=\lstname,                   % show the filename of files included with \lstinputlisting;
                                % also try caption instead of title
keywordstyle=\color{blue},          % keyword style
commentstyle=\color{dkgreen},       % comment style
stringstyle=\color{mauve},         % string literal style
escapeinside={\%*}{*)},            % if you want to add LaTeX within your code
morekeywords={*,...}               % if you want to add more keywords to the set
 }




\title{$\mathbb{MAKENOTEWITHVIMZATHURA}$}
\author{梧承}
\maketitle 
\tableofcontents

\centerline{\rule[0pt]{12.1cm}{0.05em}}
\centerline{本文仅适用于类Linux系统用户:Macos、Ubuntu等}
\centerline{\rule[5pt]{12.1cm}{0.05em}}

% ------------------------{说明部分}------------------------

\clearpage
% 章节:三点说明为什么要用Vim+Zathura+Ulsnippet进行文本输入
\section{WHY:Vim+Zathura+Ulsnippet?}

% 次章节:为什么?
\subsection{为什么?}
\begin{enumerate}
\item Vim具有极强可拓展性(什么插件都有)
\item Zathura可以实时读取pdf的变化,做到即写即阅
\item Ulsnippet极大程度降低{\LaTeX}的输写成本
		
\end{enumerate}


% 次章节:
\subsection{如何实现?}
\begin{enumerate}
		\item Vim作为编辑器编辑tex文件:		
		\item Vim安装插件:\\
			vim-auto-save $\longrightarrow$ 输入即保存\\
			latexmk $\longrightarrow$ test.tex 实时更新为 test. pdf
		\item Zathura实时读取更新的pdf
		
\end{enumerate}


% 次章节:效果预览
\subsection{效果预览}
\centering
\includegraphics[scale=0.25]{graphic1.png}

\heiti
\leftline{
	\uline{$\mathbb{NOTE}$}
	:总体还是非常简单的,让我们开始安装吧!
}

% ------------------------{安装部分}------------------------

\clearpage
% 章节:如何安装+配置Vim
\section{如何安装+配置Vim}
\centerline{|安装清单|}
\begin{itemize}
    \item Xcode $\Rightarrow$  苹果代码编译器,Homebrew前置
		\item Hombrew $\Rightarrow$ 安装工具
		\item macvim $\Rightarrow$ Vim for Mac
		\item vim-plug $\Rightarrow$ Vim插件管理工具
		\item Zathura  $\Rightarrow$ pdf阅读器
		\item Mactex $\Rightarrow$ Texlive的Mac版本
		\item latexmk  $\Rightarrow$ 编译latex的工具
		\item vim-auto-save $\Rightarrow$ vim自动保存插件
\end{itemize}

\clearpage
% 次章节:安装部分
\subsection{安装部分}

\small
\heiti
\leftline{
	\uline{$\mathbb{NOTE}$}
	:所有内容皆为复制粘贴的傻瓜式教学,但是不代表傻瓜都能安装成功。
}

% 次次章节:
\subsubsection{安装Xcode}

\begin{itemize}
		\item 打开AppleStore
		\item 搜索Xcode
		\item 安装并打开Xcode\textbf{直到创建完第一个项目}(确保完全初始化)
		
\end{itemize}

% 次次章节:安装Homebrew
\subsubsection{安装Homebrew}
\begin{itemize}
		\item 打开终端
		\item 复制以下内容并ENTER

\tiny
\begin{lstlisting}[frame=shadowbox]
/bin/bash -c "$(curl -fsSL https://raw.githubusercontent.com/Homebrew/install/master/install.sh)"
\end{lstlisting}
\end{itemize}

% 次次章节:安装mvim
\subsubsection{安装macvim}
\begin{itemize}
		\item 打开终端
		\item 复制以下内容并ENTER
\begin{lstlisting}[frame=shadowbox]
brew install macvim
\end{lstlisting}
\item 添加mvim到path:
\begin{lstlisting}[frame=shadowbox]
touch .bash_profile # 创建.bash_profile配置文件(如果已创建可忽略)
vim .bash_profile
# 按i进入输入模式并复制以下代码粘贴
alias vi=vim
alias vim=mvim
alias mvim='/usr/local/bin/mvim -v'
# 按esc退出输入模式并按:wq保存退出
# 在终端中复制以下代码启用更改后的.bash_profile
source .bash_profile
\end{lstlisting}
\end{itemize}

% 次次章节:安装vim-plug
\subsubsection{安装vim-plug}
\begin{itemize}
		\item 打开终端
		\item 复制以下内容并ENTER
\begin{lstlisting}[frame=shadowbox]
curl -fLo ~/.vim/autoload/plug.vim --create-dirs \https://raw.githubusercontent.com/junegunn/vim-plug/master/plug.vim
\end{lstlisting}

\small
\heiti
\leftline{
	\uline{$\mathbb{NOTE}$}
	:一般来说是自动直接下载plug.vim文件至:/Users/你的用户ID/.vim/autoload/plug.vim


}
\small
\heiti
\leftline{
	\uline{$\mathbb{NOTE}$}
	:如果出现在其他位置,请在.vim下创建文件夹autoload并将plug.vim拖入其中
}

\end{itemize}

% 次次章节:安装Zathura
\subsubsection{安装Zathura}
\begin{itemize}
		\item 打开终端
		\item 复制以内容并ENTER
\begin{lstlisting}[frame=shadowbox]
brew tap zegervdv/zathura
brew install zathura
brew install zathura-pdf-poppler
\end{lstlisting}
安装完成后再执行下列代码
\begin{lstlisting}[frame=shadowbox]
mkdir -p $(brew --prefix zathura)/lib/zathura
ln -s $(brew --prefix zathura-pdf-poppler)/libpdf-poppler.dylib $(brew --prefix zathura)/lib/zathura/libpdf-poppler.dylib
\end{lstlisting}
\small
\heiti
\leftline{
	\uline{$\mathbb{NOTE}$}
	:出现问题请参照以下网站
}
\begin{lstlisting}[frame=shadowbox]
https://github.com/zegervdv/homebrew-zathura#copying-to-clipboard
\end{lstlisting}
		
\end{itemize}

% 次次章节:安装Mactex
\subsubsection{安装Mactex}
\begin{itemize}
		\item 打开网站
		\item 下载Mactex并安装 
		
\end{itemize}
\begin{lstlisting}[frame=shadowbox]
http://www.tug.org/mactex/mactex-download.html
\end{lstlisting}

% 次次章节:安装latexmk
\subsubsection{安装latexmk}
\begin{itemize}
		\item 打开终端
		\item 复制代码并ENTER
		
\end{itemize}
\begin{lstlisting}[frame=shadowbox]
sudo tlmgr install latexmk 
latexmk -v
\end{lstlisting}
\small
\heiti
\leftline{
	\uline{$\mathbb{NOTE}$}
	:需要你输入密码,密码在终端中不会直接显示,这是正常现象。
}

% 次次章节:安装vim-auto-save
\subsubsection{安装vim-auto-save}
\begin{itemize}
		\item 打开终端
		\item 复制以下代码并ENTER
		\item 打开终端输入vim 
		\item 输入 :AutoSave 开启自动保存
\end{itemize}
\begin{lstlisting}[frame=shadowbox]
wget https://github.com/907th/vim-auto-save/archive/master.zip
unzip master.zip
mkdir  -p  ~/.vim/{plugin,doc,syntax}
cp vim-auto-save-master/plugin/AutoSave.vim ~/.vim/plugin/
\end{lstlisting}


\clearpage
% 次章节:配置部分
\subsection{配置部分}
\centerline{|配置清单|}
\begin{itemize}
		\item .vimrc $\Rightarrow$ vim的配置
		\item tex.snippets $\Rightarrow$ Ulsnippet的配置
		\item zathurarc $\Rightarrow$ Zathura的配置
		
\end{itemize}

% 次次章节:配置vim
\subsubsection{配置vim}
\begin{itemize}
		\item 打开终端
		\item sudo vim .vimrc
		\item 按i进入编辑模式并复制以下代码
		\item 按esc退出编辑模式并按:wq保存退出

\end{itemize}
\begin{lstlisting}[frame=shadowbox]
set nocompatible
set nu
set langmenu=zh_CN.UTF-8
set encoding=utf-8
set tabstop=2
set softtabstop=2
set shiftwidth=2
set nobackup
set mouse=a

call plug#begin('~/.vim/plugged')
Plug 'kien/ctrlp.vim'
Plug 'keelii/vim-snippets'
Plug 'sirver/ultisnips'
Plug 'vim-airline/vim-airline'
Plug 'xuhdev/vim-latex-live-preview', { 'for': 'tex' }
Plug 'morhetz/gruvbox'
Plug 'KeitaNakamura/tex-conceal.vim', {'for': 'tex'}
Plug 'wjakob/wjakob.vim'
Plug 'lervag/vimtex'
call plug#end()

filetype indent on

" vimtex配置"
let g:tex_flavor='latex'
let g:vimtex_view_method='zathura'
let g:vimtex_quickfix_mode=0

" 对中文的支持 "
let g:Tex_CompileRule_pdf = 'xelatex -synctex=1 --interaction=nonstopmode $*'
let g:vimtex_compiler_latexmk_engines = {'_':'-xelatex'}
let g:vimtex_compiler_latexrun_engines ={'_':'xelatex'}

" 设置片段隐藏 "
set conceallevel=2
let g:tex_conceal='abdmg'

" 设置对片段的支持 "
let g:UltiSnipsExpandTrigger = '<tab>'
let g:UltiSnipsJumpForwardTrigger = '<tab>'
let g:UltiSnipsJumpBackwardTrigger = '<s-tab>'

let g:livepreview_previewer = 'zathura'
let g:livepreview_engine = 'xelatex'
autocmd Filetype tex setl updatetime=1

" 设置自动保存 "
let g:auto_save = 1
let g:auto_save_events = ["InsertLeave", "TextChanged", "TextChangedI", "CursorHoldI", "CompleteDone"]
\end{lstlisting}

% 次次章节:配置vim插件
\subsubsection{配置vim插件}
\begin{itemize}
		\item 打开终端
		\item 输入 vim
		\item 输入以下代码并ENTER
		
\end{itemize}
\begin{lstlisting}[frame=shadowbox]
:PlugStatus
:PlugInstall
\end{lstlisting}

\clearpage
% 次次章节:配置Ulsnippet
\subsubsection{配置Ulsnippet}

\centerline{\rule[0pt]{12.1cm}{0.05em}}
\centerline{Ulsnippet is Magic!}
\centerline{\rule[5pt]{12.1cm}{0.05em}}
\begin{itemize}
		\item 打开终端并输入以下内容ENTER 
		\item cd ~/.vim/plugged/vim-snippets/UltiSnips
		\item touch tex.snippets
		\item 复制以下代码入tex.snippets\\
			(怎么用vim不用我再多说了吧按i复制按esc退出:wq保存)
\end{itemize}
\begin{lstlisting}[frame=shadowbox]
priority -50


# 今日日期
snippet today "Date"
`date +%F`
endsnippet

# /begin /end
snippet beg "begin{} / end{}" bA
\begin{$1}
        $0
\end{$1}
endsnippet

# 公式
snippet mk "Math" wA
$${1}$`!p
if t[2] and t[2][0] not in [',', '.', '?', '-', ' ']:
    snip.rv = ' '
else:
    snip.rv = ''
`$2
endsnippet

# 输入公式——单独成行
snippet dm "Math" wA
$$${1}$$`!p
if t[2] and t[2][0] not in [',', '.', '?', '-', ' ']:
    snip.rv = ' '
else:
    snip.rv = ''
`$2
endsnippet

# 下标自动替换
snippet '([A-Za-z])(\d)' "auto subscript" wrA
`!p snip.rv = match.group(1)`_`!p snip.rv = match.group(2)`
endsnippet
snippet '([A-Za-z])_(\d\d)' "auto subscript2" wrA
`!p snip.rv = match.group(1)`_{`!p snip.rv = match.group(2)`}
endsnippet

# 上标自动替换
snippet srr "^2" iA
^2
endsnippet
snippet cbb "^3" iA
^3
endsnippet
snippet compl "complement" iA
^{c}
endsnippet
snippet upp "superscript" iA
^{$1}$0
endsnippet

# 分数自动替换
snippet // "Fraction" iA
\\frac{$1}{$2}$0
endsnippet
snippet '((\d+)|(\d*)(\\)?([A-Za-z]+)((\^|_)(\{\d+\}|\d))*)/' "Fraction" wrA
\\frac{`!p snip.rv = match.group(1)`}{$1}$0
endsnippet

# hat、bar自动替换
snippet ht "Hat" wA
\hat{$1}$0
endsnippet
snippet br "Bar" wA
\bar{$1}$0
endsnippet

# beta、sum自动替换
snippet beta "Beta" wA
\beta
endsnippet
snippet sum "Sum" wA
\sum\limits_{$1}^{i=1$2}$0
endsnippet

# 章节自动替换
snippet sec "Section" bA
\clearpage
% 章节:$1
\section{$1}$0
endsnippet
snippet ssec "SubSection" bA
\clearpage
% 次章节:$1
\subsection{$1}$0
endsnippet
snippet sssec "SubsubSection" bA
% 次次章节:$1
\subsubsection{$1}$0
endsnippet

# 目录
snippet mulu "tableofcontent" bA
\tableofcontents
endsnippet
\end{lstlisting}

\small
\heiti
\leftline{
	\uline{$\mathbb{NOTE}$}
	:简单列举这么多,如果想要深入了解请参照以下链接
}
\begin{lstlisting}[frame=shadowbox]
https://castel.dev/post/lecture-notes-1/#inline-and-display-math
\end{lstlisting}


\clearpage
% 次次章节:配置Zathura
\subsubsection{配置Zathura}
\small
\heiti
\leftline{
	\uline{$\mathbb{NOTE}$}
	:配置的Zathura的目的在于让Zathura打开是有合适的窗口大小
}
\begin{itemize}
		\item 打开终端
		\item 复制以下代码并ENTER
		
\end{itemize}

\begin{lstlisting}[frame=shadowbox]
cd /private/etc/
touch zathurarc
sudo vim zathurarc
\end{lstlisting}
\begin{itemize}
		\item 按i进入编辑模式
		\item 复制以下代码并粘贴入zathurarc
		\item 按esc退出编辑模式并输入:wq保存退出
		
\end{itemize}
\begin{lstlisting}[frame=shadowbox]
#以宽度自适应打开
set adjust-open "width"

#字体与字号
set font "Noto Sans CJK SC Regular 10"

#GUI相关,留空可隐藏底部statusbar
set guioptions ""

#只显示文件名,否则显示完整路径
set window-title-basename true

#增强搜索,实时搜索
set incremental-search true

#显示右侧进度条
set show-v-scrollbar true

#粘贴版
set selection-clipboard clipboard

#默认高度,像素
set window-height 1300

#默认宽度
set window-width 896
\end{lstlisting}

\small
\heiti
\leftline{
	\uline{$\mathbb{NOTE}$}
	:恭喜你!你已经完成了全部安装+配置部分!
}
\small
\heiti
\leftline{
	\uline{$\mathbb{NOTE}$}
	:这是一条极其陡峭的学习曲线,但相应的也是极大的效率革命。
}

\clearpage
% 章节:使用技巧及建议
\section{使用技巧及建议}
\begin{itemize}
		\item 终端中输入vim test.tex创建文件
		\item vim中直接按$\backslash$ll 开始编译 
		\item vim中直接按$\backslash$le 查看报错
		\item 终端中输入zathura test.pdf查看pdf
		\item zathura 按q退出;按Ctrl+R反色;详细操作请百度
\end{itemize}
\small
\heiti
\leftline{
	\uline{$\mathbb{NOTE}$}
	:如果出现编译错误则不会更新pdf文件,zathur也不会更新显示
}

\end{document}



