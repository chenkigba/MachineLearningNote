\documentclass[UTF_8]{ctexart}
\usepackage{graphicx, amsfonts, setspace, color, xcolor, listings}
\begin{document}

\definecolor{dkgreen}{rgb}{0,0.6,0}
\definecolor{gray}{rgb}{0.5,0.5,0.5}
\definecolor{mauve}{rgb}{0.58,0,0.82}
\definecolor{ltgray}{HTML}{E6E6E6}
\lstset{ %
language=python,                % the language of the code
basicstyle=\footnotesize,       % the size of the fonts that are used for the code
numbers=left,                   % where to put the line-numbers
numberstyle=\tiny\color{gray},  % the style that is used for the line-numbers
stepnumber=2,                   % the step between two line-numbers. If it's 1, each line
                                % will be numbered
numbersep=5pt,                  % how far the line-numbers are from the code
backgroundcolor=\color{ltgray},  % choose the background color. You must add \usepackage{colo}
showspaces=false,               % show spaces adding particular underscores
showstringspaces=false,         % underline spaces within strings
showtabs=false,                 % show tabs within strings adding particular underscores
frame=single,                   % adds a frame around the code
rulecolor=\color{black},        % if not set, the frame-color may be changed on line-breaks wi    thin not-black text (e.g. commens (green here))
tabsize=2,                      % sets default tabsize to 2 spaces
captionpos=b,                   % sets the caption-position to bottom
breaklines=true,                % sets automatic line breaking
breakatwhitespace=false,        % sets if automatic breaks should only happen at whitespace
title=\lstname,                   % show the filename of files included with \lstinputlisting;                                 % also try caption instead of title
keywordstyle=\color{blue},          % keyword style
commentstyle=\color{dkgreen},       % comment style
stringstyle=\color{mauve},         % string literal style
escapeinside={\%*}{*)},            % if you want to add LaTeX within your code
morekeywords={*, git}               % if you want to add more keywords to the set
 }


\title{$\mathbb{HOWTOUSEGIT}$}
\author{Gusty}
\maketitle
\tableofcontents

\clearpage
% 章节:本地操作
\section{本地操作}
\centerline{|内容清单|}
\centerline{\rule[5pt]{12.1cm}{0.05em}}
\begin{itemize}
		\item 添加文件与本地提交
		\item 版本控制与管理修改
		\item 撤销修改与删除文件
\end{itemize}
\centerline{\rule[5pt]{12.1cm}{0.05em}}


% 次章节:添加文件与本地提交
\subsection{添加文件与本地提交}
\begin{lstlisting}[frame=shadowbox]
$ git init                       # 初始化git本地库

$ git add test.pdf               # 添加test.pdf到暂存区

$ git commit -m "添加test.pdf"   # 提交添加test.pdf到版本库并备注更改说明
\end{lstlisting}

% 次章节:版本控制与管理修改
\subsection{版本控制与管理修改}
\begin{lstlisting}[frame=shadowbox]
$ git log --abbrev-commit        # 查看提交历史记录 用于回到历史版本

$ git reset --hard HEAD^         # 回退到上一个版本

$ git relog                      # 查看命令历史记录 用于回到未来版本
\end{lstlisting}

% 次章节:撤销修改与删除文件
\subsection{撤销修改与删除文件}
\begin{lstlisting}[frame=shadowbox]
$ git checkout --test.pdf        # 撤销工作区的修改(至最近add或commit的状态)

$ git reset HEAD test.pdf        # 撤销缓存区的修改

$ git rm test.pdf								 # 删除版本库中的test.pdf

$ git commit -m "删除test.pdf"   # 提交删除test.pdf的请求

$ git checkout									 # 版本库文件覆盖工作区文件(一键恢复)
\end{lstlisting}


\clearpage
% 章节:远程操作
\section{远程操作}
\centerline{|内容清单|}
\centerline{\rule[5pt]{12.1cm}{0.05em}}
\begin{itemize}
		\item 关联远程库
		\item 克隆远程库
\end{itemize}
\centerline{\rule[5pt]{12.1cm}{0.05em}}

% 次章节:关联远程库
\subsection{关联远程库}
\begin{lstlisting}[frame=shadowbox]
$ git remote add origin git@github.com:YourGitID/OriginFolderName.git
# 将名为OriginFolderName的远程Repository关联到名为【origin】的分支上

$ git push -u origin master
# 首次推送【master】本地分支至【origin】远程分支上

$ git push origin master
# 推送【master】本地分支至【origin】远程分支上
\end{lstlisting}

% 次章节:克隆远程库
\subsection{克隆远程库}
\begin{lstlisting}[frame=shadowbox]
$ git clone git@github.com:YourGitID/RepositoryName.git
# 克隆远程库到当前本地文件夹
# YourGitID:你的Git的ID
# RepositoryName:远程库名称
\end{lstlisting}


\clearpage
% 章节:分支管理
\section{分支管理}


% 次章节:创建与合并分支
\subsection{创建与合并分支}
\begin{lstlisting}[frame=shadowbox]
$ git branch																	   # 查看所有分支

$ git branch test															   # 创建名为【test】的新分支

$ git switch test																 # 切换分支

$ git branch -d test												     # 删除分支

$ git merge --no--ff -m "合并分支test" test			 # 合并分支(拉齐进度)
\end{lstlisting}
			
% 次章节:分支管理策略
\subsection{分支管理策略}

\centering
\includegraphics[scale=0.6]{gitwork.png}
\begin{itemize}
		\item master:稳定正式版
		\item beta:测试版
		\item michael:michael的个人分支
		\item bob:bob的个人分支
		
\end{itemize}

\small
\heiti
\leftline{
	\uline{$\mathbb{NOTE}$}
	:我们总是在dev上测试,然后将稳定版本上传至master
}


\clearpage
% 章节:标签管理
\section{标签管理}
\small
\heiti
\leftline{
	\uline{$\mathbb{NOTE}$}
	:标签就是固定的分支;也可以理解为某个时点上的分支
}

% 次章节:创造标签和操作标签
\subsection{创造标签}
\begin{lstlisting}[frame=shadowbox]
$ git tag															  # 查看所有标签

$ git show <tagname>                    # 查看标签信息

$ git tag <tagname>							        # 为当前分支的当前版本打标签

$ git tag <tagname> <commitid>          # 为历史版本打标签

$ git log --abbrev-commit               # 查看历史commit id
\end{lstlisting}

% 次章节:操作标签
\subsection{操作标签}
\begin{lstlisting}[frame=shadowbox]
$ git tag -d <tagname>                  # 删除本地标签

$ git push origin <tagname>             # 推送本地标签

$ git push origin --tags                # 推送全部未推送过的本地标签

$ git push origin :refs/tags/<tagname>  # 删除一个远程标签
\end{lstlisting}

\clearpage
% 章节:Git Cheat Sheet
\section{Git Cheat Sheet}

\centering
\includegraphics[scale=0.4]{githelp.png}
\small
\heiti
\leftline{
	\uline{$\mathbb{NOTE}$}
	:国外友人制作的Git手册
}



\end{document}

