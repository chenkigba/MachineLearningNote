\documentclass[UTF_8]{ctexart}
\usepackage{graphicx, amsfonts, listings, color, xcolor}


\begin{document}
\definecolor{dkgreen}{rgb}{0,0.6,0}
\definecolor{gray}{rgb}{0.5,0.5,0.5}
\definecolor{mauve}{rgb}{0.58,0,0.82}
\definecolor{ltgray}{HTML}{E6E6E6}
\lstset{ %
language=python,                % the language of the code
basicstyle=\footnotesize,       % the size of the fonts that are used for the code
numbers=left,                   % where to put the line-numbers
numberstyle=\tiny\color{gray},  % the style that is used for the line-numbers
stepnumber=2,                   % the step between two line-numbers. If it's 1, each line                                 % will be numbered
numbersep=5pt,                  % how far the line-numbers are from the code
backgroundcolor=\color{ltgray},  % choose the background color. You must add \usepackage{colo}
showspaces=false,               % show spaces adding particular underscores
showstringspaces=false,         % underline spaces within strings
showtabs=false,                 % show tabs within strings adding particular underscores
frame=single,                   % adds a frame around the code
rulecolor=\color{black},        % if not set, the frame-color may be changed on line-breaks wi    thin not-black text (e.g. commens (green here))
tabsize=2,                      % sets default tabsize to 2 spaces
captionpos=b,                   % sets the caption-position to bottom
breaklines=true,                % sets automatic line breaking
breakatwhitespace=false,        % sets if automatic breaks should only happen at whitespace
title=\lstname,                   % show the filename of files included with \lstinputlisting;
                                % also try caption instead of title
keywordstyle=\color{blue},          % keyword style
commentstyle=\color{dkgreen},       % comment style
stringstyle=\color{mauve},         % string literal style
escapeinside={\%*}{*)},            % if you want to add LaTeX within your code
morekeywords={*,...}               % if you want to add more keywords to the set
  }

\title{$\mathbb{MACHINELEARNING}$}
\author{梧承}
\maketitle
\tableofcontents

\clearpage
% 章节:WEEK 1
\section{WEEK 1}

\clearpage
% 章节:WEEK 2
\section{WEEK 2}

\clearpage
% 次章节:Logistic Regression as a Neural Network
\subsection{Logistic Regression as a Neural Network}

% 次次章节:Logistic函数与Sigmoid函数
\subsubsection{Logistic函数与Sigmoid函数}
\centering
\includegraphics[scale=0.6]{sigmoidfunction.png}
\clearpage
% 次章节:Python and Vectorization
\subsection{Python and Vectorization}

\clearpage
% 次章节:Numpy的Python编程基础
\subsection{Numpy的Python编程基础}

% 次次章节:Sigmoid函数
\subsubsection{Sigmoid函数}
\begin{lstlisting}[title={$Sigmiod:s(x)=\frac{1}{(1+e^{-x})}$}][frame=shadowbox]
# GRADED FUNCTION: sigmoid

import math

def sigmoid(x):
    """
    Compute sigmoid of x.

    Arguments:
    x -- A scalar

    Return:
    s -- sigmoid(x)
    """
    
    ### START CODE HERE ### (≈ 1 line of code)
    s = 1/(1+np.exp(-x))
    ### END CODE HERE ###
    
    return s
\end{lstlisting}
\centering
\includegraphics[scale=0.6]{sigmoidfunction.png}


\clearpage
% 次次章节:Sigmoid函数的导数
\subsubsection{Sigmoid函数的导数}
\begin{lstlisting}[title=${ds=s*(1-s)}$][frame=shadowbox]
# GRADED FUNCTION: sigmoid_derivative
import numpy as np
def sigmoid_derivative(x):
    """
    Compute the gradient (also called the slope or derivative) of the sigmoid function with respect to its input x.
    You can store the output of the sigmoid function into variables and then use it to calculate the gradient.
    
    Arguments:
    x -- A scalar or numpy array

    Return:
    ds -- Your computed gradient.
    """
    
    ### START CODE HERE ### (≈ 2 lines of code)
    s = sigmoid(x)
    ds = s*(1-s)
    ### END CODE HERE ###
    
    return ds
\end{lstlisting}


\clearpage
% 次次章节:图片一维化
\subsubsection{图片一维化}

\begin{lstlisting}[title=Image 2 Vector][frame=shadowbox]
# GRADED FUNCTION: image2vector
def image2vector(image):
    """
    Argument:
    image -- a numpy array of shape (length, height, depth)
    Returns:
    v -- a vector of shape (length*height*depth, 1)
    """
    
    ### START CODE HERE ### (≈ 1 line of code)
    v = image.reshape((image.shape[0]*image.shape[1]*image.shape[2], 1))
    ### END CODE HERE ###
    
    return v
\end{lstlisting}
\centering
\includegraphics[scale=0.6]{image2vector_kiank.png}

\clearpage
% 次次章节:归一化
\subsubsection{归一化}
\begin{lstlisting}[frame=shadowbox]
# GRADED FUNCTION: normalizeRows

def normalizeRows(x):
    """
    Implement a function that normalizes each row of the matrix x (to have unit length).
    
    Argument:
    x -- A numpy matrix of shape (n, m)
    
    Returns:
    x -- The normalized (by row) numpy matrix. You are allowed to modify x.
    """
    
    ### START CODE HERE ### (≈ 2 lines of code)
    # Compute x_norm as the norm 2 of x. Use np.linalg.norm(..., ord = 2, axis = ..., keepdims = True)
    x_norm = np.linalg.norm(x ,axis = 1, keepdims = True)
    
    # Divide x by its norm.
    x = x/x_norm
    ### END CODE HERE ###

    return x
\end{lstlisting}
\centering
\includegraphics[scale=0.35]{normalizing.png}

\clearpage
% 次次章节:正向传播与Softmax函数
\subsubsection{正向传播与Softmax函数}

\begin{itemize}
	\item $z \stackrel{np.exp()}\Longrightarrow e^{z} $
	\item $ e^{z}\stackrel{/np.sum()}\Longrightarrow \frac{e^{z_i}}{\sum\limits^{3}_{i=1}} $
		
\end{itemize}
\begin{lstlisting}[title={Softmax:$S(z)=\frac{e^{z_i}}{\sum\limits^{n}_{i=1}e^{z_i}}$}][frame=shadowbox]
# GRADED FUNCTION: softmax

def softmax(x):
    """Calculates the softmax for each row of the input x.

    Your code should work for a row vector and also for matrices of shape (m,n).

    Argument:
    x -- A numpy matrix of shape (m,n)

    Returns:
    s -- A numpy matrix equal to the softmax of x, of shape (m,n)
    """
    
    ### START CODE HERE ### (≈ 3 lines of code)
    # Apply exp() element-wise to x. Use np.exp(...).
    x_exp = np.exp(x)

    # Create a vector x_sum that sums each row of x_exp. Use np.sum(..., axis = 1, keepdims = True).
    x_sum = np.sum(x_exp, axis=1, keepdims=True)
    
    # Compute softmax(x) by dividing x_exp by x_sum. It should automatically use numpy broadcasting.
    s = x_exp/x_sum

    ### END CODE HERE ###
    
    return s
\end{lstlisting}


\centering
\includegraphics[scale=0.5]{softmax.jpeg}
Softmax函数直观图

\clearpage
% 次章节:Logistic Regression with a Neural Network
\subsection{Logistic Regression with a Neural Network}
\centerline{|预处理数据的基本操作|}
\begin{itemize}
		\item 检视数据尺寸
		\item 一维化Reshape
		\item 标准化
		
\end{itemize}

\centerline{|搭建算法的主要部分|}
\begin{itemize}
		\item 
		\item 
		\item 
		
\end{itemize}

\clearpage
% 次次章节:查看数据结构
\subsubsection{查看数据结构}
\centering
\includegraphics[scale=0.42]{22.png}

% 次次章节:数据一维化
\subsubsection{数据一维化}
\small
\heiti
\leftline{
	\uline{$\mathbb{NOTE}$}
	:此处同上章Image2Vector
}
\small
\heiti
\leftline{
	\uline{$\mathbb{NOTE}$}
	:且trainsetxflatten由多个x横向拼接而成
}

\centering
\includegraphics[scale=0.42]{23.png}
\centering
\includegraphics[scale=0.48]{232.png}

使用X.reshape(X.shape[0], -1).T对矩阵进行一维化处理

\clearpage
% 次次章节:一维化的具体示例
\subsubsection{一维化的具体示例}
\small
\heiti
\leftline{
	\uline{$\mathbb{NOTE}$}
	:这是一个数据尺寸为:(2, 3, 3, 3)的图片数据
}

\centerline{\rule[0pt]{12.1cm}{0.05em}}
关于.reshape()的解释
\centerline{\rule[5pt]{12.1cm}{0.05em}}
\large
\centering
\includegraphics[scale=0.42]{245}
 $$\stackrel{.reshape(matrix.shape[0], -1)}\Downarrow$$
\centering
\includegraphics[scale=0.42]{243}
$$\stackrel{.reshape(matrix.shape[0], -1).T}\Downarrow$$
\centering
\includegraphics[scale=0.42]{244}

% 次次章节:数据标准化
\subsubsection{数据标准化}
\centering
\includegraphics[scale=0.42]{25.png}
\small
\heiti
\leftline{
	\uline{$\mathbb{NOTE}$}
	:一般来说需要除以标准差,但是图片就直接除以255好了。
}

% 次次章节:Helper Functions
\subsubsection{Helper Functions}
\centering
\includegraphics[scale=0.56]{26.png}

% 次次章节:参数初始化
\subsubsection{参数w、b初始化}
\centering
\includegraphics[scale=0.56]{27.png}
\centerline{|对np.zeros()的特别说明|}
\begin{lstlisting}[frame=shadowbox]
np.zeros( (n, m) )
# 生成一个n行m列的0矩阵
In[]:
np.zeros((3,3))
Out[]: 
array([[0., 0., 0.],
       [0., 0., 0.],
       [0., 0., 0.]])
\end{lstlisting}

\clearpage
% 次次章节:正向传播与反向传播
\subsubsection{正向传播与反向传播}
\centering
\includegraphics[scale=0.5]{28.png}


% 次次章节:优化
\subsubsection{优化}
\centering
\includegraphics[scale=0.5]{29.png}

% 次次章节:预测
\subsubsection{预测}
\centering
\includegraphics[scale=0.43]{299.png}

% 次次章节:整合模块
\subsubsection{整合模块}
\centering
\includegraphics[scale=0.37]{2999.png}

\clearpage
% 章节:WEEK 3
\section{WEKK 3}

\clearpage
% 次章节:浅层神经网络
\subsection{浅层神经网络}

\clearpage
% 次章节:Planar data classification with a hidden layer
\subsection{Planar data classification with a hidden layer}

% 次次章节:层尺寸定义
\subsubsection{层尺寸定义}
\centering
\includegraphics[scale=0.55]{31.png}

% 次次章节:参数随机初始化
\subsubsection{参数随机初始化}
\centering
\includegraphics[scale=0.55]{32.png}

% 次次章节:正向传播
\subsubsection{正向传播}
\centering
\includegraphics[scale=0.55]{33.png}
\clearpage

% 次次章节:损耗计算
\subsubsection{损耗计算}
\centering
\includegraphics[scale=0.55]{34.png}

% 次次章节:反向传播
\subsubsection{反向传播}
\centering
\includegraphics[scale=0.55]{35.png}

% 次次章节:更新参数
\subsubsection{更新参数}
\centering
\includegraphics[scale=0.55]{36.png}

% 次次章节:整合神经模块
\subsubsection{整合神经模块}

\clearpage
% 章节:WEEK 4
\section{WEEK 4}


\end{document}
